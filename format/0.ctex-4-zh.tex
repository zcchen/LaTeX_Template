% latin fonts setting :
% http://tex.stackexchange.com/questions/25249/how-do-i-use-a-particular-font-for-a-small-section-of-text-in-my-document
% 中英文搭配
% https://www.zhihu.com/question/27031612/answer/34991420
\newcommand*{\rmrfont}{\fontfamily{ppl}\selectfont}
    % 常常与楷体进行搭配使用

\usepackage[
    %fontset = none,
    zihao = -4,
    heading = true,
    linespread = 1.4,
    UTF8,
    ]{ctex}
\usepackage{titlecaps}
\Addlcwords{is for an of}

\ctexset{%
    %fontset = fandol,
    autoindent = true,
    punct = quanjiao,
    space = auto,
    today = small,
    section = {%
        %name = {\S},
        format = \large\bf,
        %titleformat = \titlecap,
        numberformat = \large\sffamily,
    },
    subsection = {%
        %name = {\S},
        format = \large\bf,
        %titleformat = \titlecap,
        numberformat = \large\sffamily,
    },
    contentsname = {\large\bf\S\ \ 目\quad 录}
}
%\ifdef\chapter{%
    %\ctexset{%
        %chapter = {%
            %name = {第,章},
            %number = \chinese{chapter},
            %format = \Huge\sffamily\heiti\centering,
            %titleformat = \titlecap,
            %aftertitle = {\bigskip\hrule height 1bp \relax
                          %\thispagestyle{zhChapterPageStyle}
                        %},
        %},
    %}
%}{%
    %section = {%
        %name = {第,节},
        %format = \Large\heiti\sffamily,
        %numberformat = \Large\sffamily,
    %},
%}


\usepackage{enumitem}
\setenumerate{%
    fullwidth,
    itemindent=\parindent,
    listparindent=\parindent,
    itemsep=0ex,
    partopsep=0pt,
    parsep=0ex,
    nolistsep
}
\setitemize{%
    fullwidth,
    itemindent=\parindent,
    listparindent=\parindent,
    itemsep=0ex,
    partopsep=0pt,
    parsep=0ex,
    nolistsep
}
